% !TEX root = Bachelorarbeit_Paul_Zilewitsch.tex
\usepackage{comment}
\usepackage{amsmath}
\usepackage{amssymb}

% deutsche Silbentrennung
\usepackage[ngerman]{babel}

% deutsche Umlauten
\usepackage[T1]{fontenc}
\usepackage{inconsolata}
\usepackage[utf8]{inputenc}

%Anführungszeichen
\usepackage[autostyle=true,german=quotes]{csquotes}

%Zeilenabstand 1,5 
\usepackage[onehalfspacing]{setspace}

%Times New Roman
\usepackage{mathptmx}
\setkomafont{disposition}{\rmfamily}

%Arrays, Tabellen und Listen
\usepackage{array}
\usepackage{tabto}
\usepackage{longtable}
\usepackage{supertabular}

%Seitenzahlen
\usepackage{scrpage2}
\cfoot[]{}
\ofoot[\pagemark]{\pagemark}
\pagestyle{scrheadings}

%Inhaltsverzeichnis
\usepackage{tocloft}
\usepackage{titletoc}
\cftsetindents{section}{0.0in}{0.5in}
\cftsetindents{subsection}{0.0in}{0.5in}
\cftsetindents{subsubsection}{0.0in}{0.5in}
\cftsetindents{paragraph}{0.0in}{0.5in}
\renewcommand{\cftsecleader}{\cftdotfill{\cftdotsep}}
\renewcommand\cftsecfont{\mdseries}
\renewcommand\cftsecpagefont{\mdseries}
\renewcaptionname{ngerman}{\contentsname}{} %kein Titel für Inhaltsv. 
%\usepackage{hyperref} %für links im inhaltsverzeichnis

%Abkürzungsverzeichnis
\usepackage{enumitem} 

%Abbildungsverzeichnis
\renewcaptionname{ngerman}{\listfigurename}{}

\titlecontents{figure}
  [0em]
  {}
  {\figurename\enspace\thecontentslabel:\enspace}
  {}
  {\titlerule*[1pc]{.}\contentspage}
  
  %Tabellenverzeichnis
\renewcaptionname{ngerman}{\listtablename}{}

\titlecontents{table}
  [0em]
  {}
  {\tablename\enspace\thecontentslabel:\enspace}
  {}
  {\titlerule*[1pc]{.}\contentspage}

%Seitenränder
\usepackage{geometry}
\geometry{a4paper, top=21mm, left=30mm, right=20mm, bottom=20mm,
headsep=10mm, footskip=10mm}

%C# Code-Anzeige
\usepackage{color}
\definecolor{bluekeywords}{rgb}{0.13,0.13,1}
\definecolor{greencomments}{rgb}{0,0.5,0}
\definecolor{redstrings}{rgb}{0.9,0,0}

\usepackage{listings}
\lstset{language=[Sharp]C,
  showspaces=false,
  showtabs=false,
  breaklines=true,
  showstringspaces=false,
  breakatwhitespace=true,
  escapeinside={(*@}{@*)},
  commentstyle=\color{greencomments},
  keywordstyle=\color{bluekeywords},
  stringstyle=\color{redstrings},
  basicstyle=\ttfamily
}

%Überschriften Definition
\usepackage{titlesec}

\titleformat{\section}
  {\normalfont\fontsize{16}{17}\rmfamily\bfseries}
  {\thesection}
  {2em}
  {}

\titleformat{\subsection}
  {\normalfont\fontsize{14}{17}\rmfamily\bfseries}
  {\thesubsection}
  {1.5em}
  {}
  
  \titleformat{\subsubsection}
  {\normalfont\fontsize{14}{17}\rmfamily\mdseries}
  {\thesubsubsection}
  {0.75em}
  {}
\titlespacing{\section}{0pt}{*6}{*5} %{Einzug} {abstand nach oben}{abstand nach unten}
\titlespacing{\subsection}{0pt}{*5}{*3}
\titlespacing{\susubbsection}{0pt}{*4}{*2}

%Fußnoten
\usepackage[multiple, hang]{footmisc} % Komma zwischen mehreren Fußnoten
\setlength{\footnotemargin}{1em}
%\usepackage[justification=RaggedRight, singlelinecheck=false]{caption}
%\usepackage{caption}
%\captionsetup{%
%  textfont=footnotesize,
%  labelfont=footnotesize,
%  font=singlespacing,
%}


%Bilder
\usepackage{graphicx}


%Quellen
%\usepackage{biblatex}
\usepackage[square, numbers]{natbib}
\usepackage[overridenumbers]{bibtopic}
%\usepackage[numbers,round]{natbib}


%Farben und Zeichnungen
\usepackage{pict2e}  
\usepackage{color}
\usepackage{ amssymb }
\definecolor{gray}{RGB}{150, 150, 150}



