% !TEX root = Bachelorarbeit_Paul_Zilewitsch.tex
\section{Service Desk nach ITIL v3}

\subsection{Begriffsabgrenzung}

\noindent Für die Klärung des Begriffs \enquote{Service Desk} ist es sinnvoll, sich auf die Information Technology Infrastructure Libary - kurz ITIL - zu beziehen.
ITIL ist zwar keine Norm, die in der IT-Branche eingehalten werden muss, dennoch bezieht man sich im IT-Service Management ausschließlich auf ITIL.
Bereits 1989 wurde die Central Computer and Telecommunication Agency (CCTA) von der britischen Regierung beauftragt, Geschäftsprozesse und ihre Abhängigkeiten zu beschreiben und schriftlich festzuhalten.\footnote{Vgl. Olbrich, A. (2008): ITIL kompakt und verständlich, S.1.}
Ziel war es, Abläufe in der Unternehmenswelt darzustellen und dadurch die IT-Betriebskosten zu reduzieren. Im Laufe der Jahre wurden die ersten Ausarbeitungen überarbeitet und ergänzt. Die ITIL Edition 2011 ist die derzeitig neuste Fassung und stellt ein Update der 2007 veröffentlichten Version ITIL v3 dar.\footnote{Vgl. Vorlesung ITIL, S.25 5.Semester.}
Auch bestimmte Normen leiten sich aus dem ITIL-Rahmenkonzept ab. Der internationale Standard ISO/IEC 20000 beispielsweise basiert auf die Version ITIL v2 und definiert die Minimalanforderungen des IT-Service-Managements für Organisationen. \footnote{Vgl. Buchsein, R./Victor, F./Günther, H./Machmeier, V. (2008): IT-Management mit ITIL® V3, S. 5ff.}
ITIL kann deshalb als "Quasi-Standard" gesehen werden. Es ist ein  Best-Practice Leitfaden, der beschreibt "Was zu tun ist, aber nicht wie". Das macht deutlich, dass ITIL durchaus Handlungs -und Interpretationsspielraum zulässt, aber dennoch in einem vorher definierten Rahmen greifbar sein muss. Beschrieben wird ITIL v3 in fünf Büchern, auf die später noch eingegangen wird:

\begin{itemize}
\item Service Strategy
\item Service Design
\item Service Transition
\item Service Operation
\item Continual Service Improvement
\end{itemize}

\noindent
ITIL kann als Framework gesehen werden, mit dem Abläufe im Bereich IT Service beschrieben werden können. Genauer gesagt, spricht man IT Services Management, kurz ITSM. Hier werden alle Methoden erläutert, die nötig sind, um die bestmögliche Unterstützung von Geschäftsprozessen durch die IT-Organisation zu erreichen.\footnote{Vgl. Ebel ,N. (2008):  ITIL® V3 Basis-Zertifizierung, S.27 ff.} In ITIL ist ein Prozess ist definiert als \enquote{Satz von in Wechselbeziehung oder Wechselwirkung stehenden Tätigkeiten (und Mitteln), die Eingaben in Ergebnisse umwandelt. Zu den Mitteln können Personal, Einrichtungen und Anlagen, Technologie und Methodologie gehören. Eingaben für einen Prozess sind üblicherweise  Ergebnisse anderer Prozesse.}\footnote{Müller, A. (2015): Vorlesung ITIL, Skript S.18, 5.Semester.}\footnote{DIN EN ISO 9000:2005, Qualitätsmanagementsysteme , S. 23.}

\noindent
Jeder IT Service Management-Prozess hat eine charakteristische Zielrichtung und wird durch Funktionen unterstützt. Eine Funktion besteht aus einer Gruppe von Personen und deren Werkzeuge, die dafür verwendet werden, ein oder mehrere Prozesse oder Aktivitäten zu stützen.\footnote{ Vgl. Cannon, D./Wheeldon, D. (2007): Service Operation, S. 233.} Außerdem bewirkt das Zusammenspiel verschiedener IT Service Management-Prozesse, dass dem Kunden die notwendigen IT Services zur wirkungsvollen Unterstützung seiner Geschäftsprozesse geliefert werden. Der Service Lifecycle in Abbildung \ref{fig:ITIL_Lebenyzyklus} veranschaulicht genau diese Kernprozesse und Kernfunktionen, die ein IT-Prozess während seiner gesamten Lebensdauer besitzt. Die einzelnen Teilbereiche decken sich mit den zuvor aufgeführten Büchern von ITIL v3.\footnote{Vgl. Ebel,N. (2008):  ITIL® V3 Basis-Zertifizierung, S.27 ff.}

\begin{figure}[h!]
\centering
	\includegraphics[width=0.50\textwidth]{Abbildungen/ITIL_Lebenszyklus}
	\caption[ITIL Service Lifecycle]{ITIL Service Lifecycle, Quelle: http://os.itil.org/osMedia/site/t1 
	media/JPEG/01\_itil\_imap.jpg}
	\label{fig:ITIL_Lebenyzyklus}
\end{figure}

\noindent
Auf alle Teilbeibereiche einzugehen, wäre zu zeitaufwendig, ist aber auch gar nicht nötig. Der Service Desk ist nämlich Bestandteil der Service Operation und somit die richtige Anlaufstelle für die Begriffsklärung. \newline \enquote{Service Operation beschreibt den Abschnitt des Lebenszyklus, der von den Kunden primär wahrgenommen wird.}\footnote{Vgl. Ebel, N. (2008): ITIL® V3 Basis-Zertifizierung, S.439.} In der Service Operations-Phase werden die Prozesse und Funktionen beschrieben, die einen stabilen  und bestmöglich IT Service garantieren sollen. Bei dieser Verbindung von IT Organisation und Kunde wird besonders auf den Kunden eingegangen. Der Service Desk ist hierbei \enquote{die zentrale Anlaufstelle, der Single Point of Contact (SPoC) zwischen Anwender und der IT-Organisation}\footnote{Vgl. Ebel, N. (2008): ITIL® V3 Basis-Zertifizierung, S.439.}. Wie der Name schon verrät, kommt der Anwender nur über diese Schnittstelle in Kontakt mit der IT. Hier werden Meldungen der Anwender üblicherweise erfasst, kategorisiert und eingetragen. Der Service Desk ist nicht nur eine Kommunikationsunterstürzung, sondern bietet gleichzeitig eine Auskunft für bereits bekannte Probleme. Dadurch kann bei häufig auftretenden Service-Unterbrechungen schneller gehandelt werden. Auch Supportanfragen, Beschwerden, Verbesserungsvorschläge oder Änderungswünsche können in den Service Desk eingetragen werden. Diese einzelnen Managementbereiche von ITIL v3 (Incident -, Problem -, Configuration -, Change -und Release Management) laufen im Service Desk zusammen, sodass der Agent  nicht mehr entscheiden muss, in welchen Bereich sein Problem/Vorfall eingeordnet werden muss. Das ist dann Aufgabe des Supports. Die nachfolgende Abbildung verdeutlicht dieses Vorgehen.

\begin{figure}[h!]
\centering
	\includegraphics[width=0.75\textwidth]{Abbildungen/SPOC_2.png}
	\caption[Single Point of Contact]{Single Point of Contact, Quelle:http://edoc.hu-berlin.de/
	conferences/dfn2006/fischlin-roger-105/PDF/fischlin.pdf}
	\label{fig:ITIL_Lebenyzyklus}
\end{figure}


\colorbox{red}{Einzelnen Bereiche kurz erklären?}

\subsection{Unterschied zu Help Desk}

\noindent
Bei der Begriffsabgrenzung zwischen Help Desk und Service Desk ist Vorsicht geboten. In mehreren Quellen ist zu finden, dass Help Desk (auch User-Help-Desk) lediglich ein veralteter Begriff für den Service Desk sei.\footnote{Vgl. Victor, F./Günther, H. (2005): Optimiertes
IT-Management mit ITIL, S.24.}\footnote{Meier, A./Myrach, T, (2004): IT-Servicemanagement, S.26.} Im Internet heißt es Beispielsweise auf Wikipedia: \enquote{Der Artikel Help desk und Service Desk überschneiden sich thematisch.}\footnote{https://de.wikipedia.org/wiki/Servicedesk, Stand 08.11.2013} In anderen Literaturquellen tritt der Begriff Help desk erst gar nicht auf oder wird dem Service Desk gleichgesetzt\footnote{Vgl. Olbrich, A. (2008): ITIL Kompakt und verständlich, S.19.}. Im Zweifelsfall sollte man sich direkt auf das Buch ITIL Service Operation beziehen. In dem heißt es übersetzt unter dem Stichwort Help Desk:
\enquote{Eine Anlaufstelle für Anwender, um Incidents zu erfassen. Ein Help Desk ist in der
Regel eher technisch orientiert als ein Service Desk und stellt keinen Single Point
of Contact für die gesamte Interaktion bereit. Der Begriff „Help Desk“ wird häufig
auch als Synonym für Service Desk verwendet.} \footnote{Vgl. Cannon, D./Wheeldon, D. (2007): Service Operation, S. 233. Übersetzung entnommen aus: Ebel, N. (2008): ITIL® V3 Basis-Zertifizierung, S. 699.} \newline
In den weiteren Ausführungen gilt deshalb der Help Desk als Synonym für den Service Desk.



\subsection{Anforderungen eines Service Desk}

\subsubsection{Aufgaben}
\noindent
Im folgenden sollen die wichtigsten Aufgaben eines Service Desk aus Sicht der ITIL v3 erläutert werden. Hierfür wird zunächst stichpunktartig die Kernaussage festgehalten, um sie anschließend zu erläutern. Dabei beziehen sich die Kernaussagen auf der Ausarbeitung Olbrichs aus \enquote{ITIL Kompakt und verständlich}
\footnote{Olrich, A. (2008): ITIL kompakt und verständlich, S.19 f.}
und sind eine leicht abgewandelte Form vom ITIL v3 Band Service Operation \footnote{Vgl. Cannon, D./Wheeldon, D. (2007): Service Operation, S. 110.}

\begin{itemize}
\item \enquote{Einheitlichen, zentralen Kommunikationsschnittstelle (SPoC) mit konkreten Ansprechpartner}
\end{itemize}
\noindent
Der Kunde hat mehrere Möglichkeiten den Support zu kontaktieren. Schreibt der Kunde eine E-Mail an den Support, könnte diese ausgewertet und den Service Desk eingetragen werden. Ebenso könnte er selbst einen Eintrag über ein Web-Frontend erstellen. Oder aber der Support erstellt einen solchen Eintrag im Service Desk, wenn der Kunde zum Telefon greift. Es ist aber Voraussetzung, dass eine einheitliche und zentrale Kommunikationsschnittstelle bereitgestellt wird.


\begin{itemize}
\item \enquote{Aufnahme, Dokumentation und Auswertung aller Vorfälle}
\end{itemize}
\noindent
Wenn alle Vorfälle ordnungsgemäß aufgenommen und dokumentiert wurden, kann schneller reagiert werden, wenn sich ein Problem wiederholt. Dass alle Vorfälle auch ausgewertet werden sollten, ist verständlich und kann eventuell dazu beitragen, Folgeprobleme frühzeitig zu erkennen.

\begin{itemize}
\item \enquote{Überwachung, Nachverfolgung und Eskalation von laufenden
Supportvorgängen. Frühzeitiges Erkennen von
Bedürfnissen und Problemsituationen}
\end{itemize}
\noindent
Wie im Punkt zuvor erwähnt können Probleme frühzeitig erkannt werden, indem Vorfälle genauestens ausgewertet werden. Das ist aber längst nicht die einzige Möglichkeit, Bedürfnisse der Kunden zu erkennen. Gute Mittel für vorausschauende Handlungen sind  bspw. Monitoring-Systeme oder log-Files. Sie liefern technische Informationen, die - nach einer Auswertung - Aufschluss über die aktuelle Lage des Kunden geben und in den Service Desk integriert werden könnten.


\begin{itemize}
\item \enquote{Überprüfung der Einhaltung des Dienstleistungsgegenstands
anhand von Service-Level-Agreements}
\end{itemize}
\noindent
Mithilfe des Service Desks kann kontrolliert werden, ob die vereinbarten Leistungen zwischen Auftraggeber und Beauftragter eingehalten wurden, wenn alle Vorfälle dokumentiert wurden.

\begin{itemize}
\item \enquote{Reporting – Beauskunftung gegenüber den Usern (Kunden)
und dem Management. Informationen über den aktuellen
Status von Vorgängen, geplanten Änderungen und
verschiedenen Nutzungsmöglichkeiten}
\end{itemize}
\noindent
Der Service Desk dient außerdem dazu, stets mit dem Kunden im Kontakt zu stehen. So können Information an den Kunden weitergeleitet oder auf der anderen Seite aktuelle Vorgänge, Status etc. des Kunden verfolgt und analysiert werden. 

\begin{itemize}
\item \enquote{Überprüfen der Kundenzufriedenheit, Stärkung der Kundenbeziehung.
Kontaktpflege. Aufspüren neuer Geschäftschancen}
\end{itemize}
\noindent
Nicht zuletzt kann der Service Desk auch als Instrument für einen ständigen Kontakt zum Kunden eingesetzt werden. Der Kunde hat dadurch den Eindruck, permanent mit der Support und somit der Firma verbunden zu sein. Das kann das Verhältnis zum Kunden stärken oder gar neue Geschäftsmöglichkeiten eröffnen.\\


\subsubsection{Kategorien}

\noindent
Grundsätzlich gibt es drei Kategorien bezüglich der Architektur eines Service Desks:

\begin{itemize}
\item Lokaler Service Desk
\item Zentraler Service Desk
\item Virtueller Service Desk
\end{itemize}

\noindent
Der lokale Service Desk zeichnet sich dadurch aus, dass er innerhalb eines Bereiches selbständig agiert. Mehrere Unternehmensstandorte oder verschiedene Bereiche eines Unternehmens haben jeweils einen eigenen Service Desk. Dieser kann präzise auf die Probleme und Prozesse vor Ort angepasst werden. Jedoch gestaltet sich eine Zusammenarbeit mehrere Bereiche gerade wegen dieser individuellen Service Desks schwierig. \\

\noindent
Beim zentralen Service Desk ist diese bereichsübergreifende Zusammenarbeit keine Hürde, da es einen Service Desk gibt, der für alle Bereiche eine gleichmäßige Zuständigkeit besitzt. Die Prozesse und Abläufe aller Benutzer sind hier identisch. Zwar sind die Kommunikationsmöglichkeiten hier sehr umfangreich, jedoch wächst die Informationsmenge rasant an. Eine kundennahe Betreuung wird durch den hohen Organisationsumfang deutlich erschwert. \\

\noindent
Bei der Kompromisslösung dieser beiden Architekturmodellen stößt man auf den virtuellen Service Desk. Die Informationseingabe der Benutzer kann durchaus an verschiedenen Standorten erfolgen ganz wie beim lokalen Service Desk. Doch alle Daten werden gesammelt und zentral verwaltet, was dem zentralen Service Desk entspricht. Entscheidend ist hierbei, dass es einheitliche Prozesse und Abläufe an den einzelnen Standorten geben muss. Individuelle Service Desks wären zu aufwendig in der Verwaltung. Auch so ist der Ressourcen -und Organisationsaufwand gegenüber den anderen Modellen enorm und bedarf guter Planung. \footnote{Vgl. Olbrich, A. (2008): ITIL kompakt und verständlich, S.21.} \footnote{Vgl. Cannon, D./Wheeldon, D. (2007): Service Operation, S. 111 f.} \\


\subsection{Analyse verschiedener Service Desk-Lösungen}

\subsubsection{Schwerpunkte der Analyse und Ausgewählte Service Desk-Lösungen}

\noindent
Nach diesem sehr theoretischem Ansatz die Anforderungen eines Service Desks zu klären, wird nun auf Beispiele in der Praxis eingegangen. Wichtig sind dabei nicht Kriterien wie das äußere Erscheinungsbild oder die Kosten. Ziel soll es sein durch einen Vergleich gängiger Softwarelösungen, Verbesserungsmöglichkeiten der eigene Service Desk-Funktionalität in GEBman10 zu ermitteln. Dabei wird auf die drei folgenden Punkte Wert gelegt:

\begin{itemize}
\item Funktionalität:\\
		Die Funktionalität ist wohl das fundamentalste Kriterium. Hier ist entscheidend, welche 			
		Möglichkeiten dem Benutzer gegeben werden bspw. Meldungen/Tickets anzulegen, zu 
		zuweisen, zu suchen oder zu filtern. Wichtig ist aber auch, welche Informationen in welcher 
		Darstellungsform erhalten sind (Diagramme etc.) und welche Daten erfasst werden müssen 
		bzw. können.\\
		 
\item Bedienbarkeit:\\
		Aus diesem Blickwinkel wird untersucht, welche Bedienelemente zur Verfügung stehen. Eine	
		Bewertung nach intuitiver Bedienbarkeit ist schwierig vorzunehmen, da das immer eine
		subjektive Ansicht enthält.\\
		
\item Anpassbarkeit:\\
		Inwieweit kann man bspw. die grafische Oberfläche vom Benutzer geändert und auf die 
		eigenen Bedürfnisse angepasst werden.\\		
\end{itemize}


%hier sollte noch ein abschließender Satz hin

\noindent
Nach den Recherchen auf mehreren Review und Ranking Websites zum Thema Service Desk / Help Desk, sind drei Softwarelösungen wiederholt erwähnt und gut bis sehr gut bewertet worden.\footnote{siehe Literaturverzeichnis}Diese drei webbasierten Anwendungen werden nun vorgestellt und anschließend ihre Stärken bzw. Besonderheiten dargelegt.

\begin{itemize}
\item Freshdesk:\\
		Girish Mathrubootham beschloss 2010 nach dem Lesen eines Nachrichtenartikels die Firma 
		Freshdesk ins Leben zu rufen. Das gleichnamige Produkt wird laut eigenen Angaben von rund 
		70.000 Kunden aller Unternehmensgrößen genutzt.\footnote{http://freshdesk.de/kunden/}
		\\
		 
\item Desk.com:\\
		Das Unternehmen Salesforce.com legt laut eigenen Angaben großen Wert auf mobile 
		Endgeräte und soziale Netzwerke. \footnote{http://www.salesforce.com/de/company/
		newspress/press-releases/2012/02/120201.jsp}Die Service Desk -Lösung des Unternehmens  
		nennt sich 	Desk.com und zu ihren Kunden zählen unter anderem die Firma FlixBus und die 
		Commerzbank.
		\\
		
\item Zendesk:\\
		Zendesk beschreibt sich selbst als \enquote{Kundenservice-Plattform}. Die gleichnamigen 
		Firma hat nach eigenen Angaben mehr als 75 000 Unternehmen, die diese Plattform nutzen. 
		Entstanden ist das Unternehmen 2007 aus einer Idee von drei Freunden aus Kopenhagen.
		\footnote{https://www.zendesk.de/about/}
		\\		
\end{itemize}

\noindent
Aktuell benutzt der Support von der KMS Computer GmbH die Software SysAid für den Service Desk. Durch eine Vielzahl von Einträgen und Erfahrungen der Mitarbeiter im Support, ist es sinnvoll auch diese Anwendung mit in den Vergleich einfließen zu lassen.

\begin{itemize}
\item SysAid:\\
		 Die Help Desk-Software SysAid wird nach eigenen Angaben in über 10 000 Unternehmen in 
		 140 Ländern eingesetzt. Die Firma SysAid Technologies wurde 2002 gegründet und zählt somit 
		 zu den erfahrenden Unternehmen dieser Branche.
		\\
\end{itemize}	


\subsubsection{Freshdesk}
\noindent
Alle im Freshdesk angemeldeten Support Mitarbeiter werden als Agenten bezeichnet
Der Freshdesk besticht mit seinem sehr simplen Dashboard, was der Bedienbarkeit zugutekommt. Zur Erklärung: ein Dashboard ist üblicherweise eine kompakte meist grafisch aufbereitete Ansicht von Informationen.\footnote{Vgl. https://help.salesforce.com/HTViewHelpDoc?id=bi\_dashboard.htm\&language=de} Der Agent erhält hier nur die wichtigsten bzw. neuesten Informationen. Auf den ersten Blick kann der Agent sehen, welche Tickets offen, nicht zugewiesen oder überfällig sind. Zusätzlich erhält der Agent die Möglichkeit, individuelle Aufgaben zu notieren und wie eine Checkliste abzuarbeiten. Ein sehr nützliches Feature für kleinere Notizen bzw. Probleme.\\ 
Erst im zweiten Menüpunkt kann der Agent Tickets filtern. Hierfür gibt es gängige auswählbare Filter oder die Option einen Filter selbst zu konfigurieren und zu speichern. Wählt der Agent hier nun ein oder mehrere Ticket aus, gelangt er in eine Detailansicht eines Tickets und kann dann zwischen den ausgewählten wechseln. In der Detailansicht kann er nun den gesamten Verlauf betrachten, auf das Ticket antworten, das Ticket weiterleiten etc. . Ein Seitenmenü ermöglicht die Einsicht auf die wichtigsten Informationen des Ticketerstellers. Außerdem können die Ticket-Eigenschaften in diesem Seitenmenü geändert und somit die Priorität und der Status festgelegt werden. Zusätzlich gibt es eine Typisierung des Tickets. Im Punkt 2.1 wurden die verschiedenen IT Service Managementbereiche dargelegt. Durch die Typisierung kann festgelegt werden, welchem Managementbereich das Ticket entspricht. Handelt es sich beispielsweise um einen Notfall, wird der Typ als Incident bestimmt und kann so bei der Suche schneller gefunden werden.\\
Ein weiterer Menüpunkt nennt sich \enquote{Soziales} und hat eine spezielle Funktionalität zu bieten. Freshdesk bieten nämlich die Möglichkeit, die Kunden über einen \enquote{sozialen Support} zu betreuen. Laut eigenen Angaben begründet Freshdesk den Kontakt mit Kunden über soziale Kanäle wie folgt: \enquote{Tatsächlich erwarten 32 \% der Kunden in sozialen Netzwerken eine Antwort auf Ihre Anfragen innerhalb von 30 Minuten.} weiterhin heißt es, bei schnellen und effektiven Antworten würden 71\% der Kunden den Support weiterempfehlen. Wie Freshdesk zu diesen Zahlen kommt, bleibt zunächst unklar. Recherchiert man ein wenig im Netz, so kann man diese Zahlen in einem Bericht von t3n wiederfinden, der auf einer Studie von  Bain \& Company beruht.\footnote{Vgl. http://t3n.de/news/zufriedene-kunden-geht-support-549532/} Wichtiger als die Zahlen ist jedoch die Idee, über soziale Netzwerke mit dem Kunden zu interagieren und somit eine teils freundschaftliche Beziehung aufzubauen. Erlaubt man Freshdesk im Administrator-Bereich, sich über ein Twitter-Konto einzuloggen, so kann direkt im Service Desk auf Twitter zugegriffen werden. Auch eine Facebook-Seite kann in den Service Desk von Freshdesk integriert werden. Diese Funktionalität wirkt sehr modern und bietet neue Ansatzpunkte in der Kundenbetreuung.\\
Im dritten Menüpunkt erhält der Agent die Option eine Knowledge Base anzulegen. Diese Wissensdatenbank kann direkt vom Kunden aufgerufen werde, um eine erste Hilfestellung bei bekannten Problemen zu erhalten. Mit Kategorien wie Frequently Asked Questions (FAQs) kann die Wissensdatenbank unterteilt werden und bietet dem Kunden so eine gute Übersicht.\\
Auch ein Forum kann mit dem Freshdesk gepflegt werden. Im Unterpunkt Foren können mehrere Foren verwaltet werden und Themen wie \enquote{Tips und Tricks} oder \enquote{Wie erstelle ich ein Ticket?} dem Kunden ein optimales Handbuch oder Nachschlagewerk liefern.\\
Des Weiteren kann im Menüpunkt Berichte auf umfassende Analysen Einsicht genommen werden. Nicht nur wie lange ein Ticket durchschnittlich bearbeitet wurde ist in Diagrammen dargestellt, sondern auch wie viele Tickets die Kunden aufgegeben haben oder wie viel Tickets ein Agent schon bearbeitet hat. Trotz der großen Informationswiedergabe bleibt Freshdesk übersichtlich und gut strukturiert. Die Berichte kann sich der Agent auch per Mail in Form einer PDF - oder CSV-Datei zuschicken lassen.\\
Der bereits erwähnte Administrator-Bereich kann sehr gut genutzt werden, um den Service Desk anzupassen. Von allgemeinen Einstellungen wie den Feldern, die bei der Ticketerstellung ausgefüllt werden müssen, bis zu dem Import von Daten aus anderen Service Desk-Lösungen kann der Freshdesk sehr gut and die Bedürfnisse der Agents oder Gruppen von Agents angepasst werden.\\
Egal in welchem Menüpunkt der Agent sich bewegt, es stehen ihm immer ein Button für das Anlegen eines neuen Tickets und einer Suche in der oberhalb liegenden Menüleiste zur Verfügung. Der Freshdesk ist somit ein in sich schlüssiges System mit vielen modernen und anschaulichen Extras.\\\\


\subsubsection{Desk.com}
\noindent
Desk.com ist eine Service Desk-Lösungen, die besonderen Wert auf den mobilen Einsatz des Supports legt. Das spiegelt sich auch in der Desktop Webanwendung wieder. Die Oberfläche erinnert stark an eine App auf einen mobilen Endgerät. Die Benutzer werden in Desk.com ebenfalls Agents genannt. 
Eine statische Menüleiste ist auch hier oberhalb der Ansicht zu finden. Hier befinden sich große Buttons für das Anlegen eines neuen Tickets, eine Suche, ein Button für weitere Menüs und eine Art Tab-Ansicht der neusten Tickets. Das Dashboard besteht aus eine Auflistung aller Tickets, die nur die nötigsten Informationen liefern. Spalten können sich aber noch zusätzlich einblenden lassen. Filtern lassen sich die Tickets an dieser Stelle mit einer Auswahl auf der linken Seite. Jedoch ist diese Auswahl sehr eingeschränkt auf alle Tickets oder Tickets, die dem Agent zugewiesen wurden. Die Service Desk-Lösung von Desk.com hat ebenfalls einen eigenen Menüpunkt für die Einsicht von Berichten. Nicht ganz so umfangreich wie beim Freshdesk, aber dennoch anschaulich in Diagrammen dargestellt. Die Funktionalität einer Wissensdatenbank bietet der Desk.com ebenso. Allerdings kann diese nur im Administrator-Bereich verwaltet und erweitert werden.\\
Insgesamt wirkt der Desk.com maßgeschneidert für mobile Endgeräte. Und genau hier liegen auch die Stärken der Service Desk-Lösung. Durch die großen Bedienelemente und stark vereinfachten Ansichten, wird das Arbeiten auf Tablet o.ä. deutlich erleichtert. Für eine Desktop Variante nicht unbedingt die beste Wahl, auch weil sich der Service Desk - wenn überhaupt - nur sehr umständlich auf die individuellen Bedürfnisse anpassen  lässt. Für Mitarbeiter, die ständig unterwegs sind, ist das System durchaus attraktiv.\\\\


\subsubsection{Zendesk}
\noindent 
Der Zendesk hat seine Stärken in den umfassenden Hilfestellungen für die Agents. Schon im Dashboard erhält ein Agent eine knappe aber präzise Erklärung der einzelnen Teilbereiche des Service Desk. Zunächst muss der Agent die einzelnen Kanäle wie E-Mail oder Telefon einrichten. Auch im Zendesk sind  Twitter und Facebook als Kommunikationswege denkbar. In fast allen Einrichtungsschritten wird der Agent mit Anweisungen unterstützt und kann sich direkt im Zendesk ein Video-Tutorial anschauen.\\
Eine statische Leiste befindet sich im Zendesk oberhalb mit einer Suchfunktion und er Möglichkeit neue Tickets zu erstellen. Ein ebenfalls festes Menü am Seitenrand hat nur die wichtigsten Unterpunkte: Dashboard, Tickets , Berichte und Einstellungen. Eine Anordnung, die bereits aus den anderen Service Desk-Lösungen bekannt ist. Das Menü kann angepasst werden, wenn alle Kanäle eingerichtet sind.\\
Neben den bereits bekannten Features von den andern Softwarelösungen, bietet Zendesk weitere Funktionalitäten. Durch eine große Auswahl an Apps kann der Service Desk ganz nach eigenen Vorstellungen des Agents angepasst werden. Ein einfaches Beispiel hierfür ist das Anzeigen von Kontaktinformationen direkt neben einem Ticket. Die App trägt den schlichte Namen Benutzerdaten. Wurde diese Erweiterung erfolgreich installiert, kann der Agent über einen Button die Funktionalitäten der Apps bei der Ticketübersicht nutzen,  um weitere Informationen über den Kunden zu gewinnen. Dabei besteht die Möglichkeit Notizen oder Details über den Kunden in der App einzutragen.\\
Über ein Web Widget ist es möglich, auf Komponenten von Zendesk wie die Wissensdatenbank oder Live-Chat zu zugreifen. Als Widget bezeichnet man einfache kleine clientseitige Programme, die durch minimalem Eingabeaufwand zusätzliche Funktionen oder Informationen bereitstellen.\footnote{http://www.onlinemarketing-praxis.de/glossar/widget} \footnote{https://www.w3.org/TR/widgets/\#introduction}  Dieses Web Widget kann in Webseiten eingebettet werden, in dem in den Einstellungen das Widget aktiviert und auf der Webseite der Source-Code eingebunden wird. Eine Funktionalität die vor allem bei Unternehmenswebseiten  eingebaut werden könnte.\\
Um festzustellen, wie sehr die Kunden mit dem Service Desk zufrieden sind, kann mit dem Zendesk eine Kundenumfrage gestartet werden. Die Fragen hierfür lassen sich allerdings nicht konfigurieren. Der Kunde kann (bei aktivierter Option) auch die Tickets bewerten und somit ebenfalls Feedback für den Agent geben.\\
Zendesk bietet eine Vielzahl von Funktionalitäten, die sowohl neue Kommunikationsmöglichkeiten mit dem Kunden garantieren, als auch die Bedienung des Service Desk für die Agents erleichtern. Es ist aber anzumerken, dass es durchaus eine gewisse Zeit in Anspruch nimmt, alle Features einzurichten und richtig zu bedienen. Einsteiger Agents sollten deshalb gut geschult werden.\\\\


\subsubsection{SysAid}
\noindent
Da diese Bachelorarbeit in einem knappen Zeitrahmen fertiggestellt werden muss, konnten keine ausführlichen Betrachtungen der Service Desk-Lösungen durchgeführt werden, wenn diese mit vielen Daten gefüllt sind. Hierzu wären mehrere Kontaktinformationen und Ticket-Erstellungen nötig. Deshalb wird nun der firmeneigene Support nach Besonderheiten in der Softwarelösung SysAid befragt. Bei täglichem Gebrauch kommen unvorhersehbare Situationen zusammen, die kaum in der Vorbetrachtung zu erahnen sind. Daher ist die Betrachtung vom SysAid der KMS Computer GmbH eine sinnvolle Vorgehensweise bei der Ermittlung von Verbesserungsmöglichkeiten im Bereich Service Desk-Anwendungen.\\\\




\subsubsection{Fazit der Analyse}

\noindent
Alle Service Desk-Systeme, die betrachtet wurden, waren auf ihre Weise individuell. Die Schwerpunkte waren unterschiedlich gelegt und einen "Sieger" der Analyse zu bestimmen wäre daher nicht sinnvoll. Wichtiger ist viel mehr, wie die Stärken der Systeme möglicherweise Anreize für Verbesserungen anderer Service Desk-Lösungen bieten.\\
Der Freshdesk  \\\\

\noindent
Es lassen sich aber auch Gemeinsamkeiten benennen, die sich in allen Systemen in leicht abgewandelter Form wiederfanden. 

\begin{itemize}
\item Menüleiste ermöglichte es dem Agent/Benutzer jederzeit zu suchen und ein neues Ticket anzulegen.
		 
\item durch farbliche Kennzeichnung waren die Status der Tickets sofort einsehbar
		
\item  Kontaktdaten vom Kunden in der Detailansicht eines Tickets

\item Filtereinstellungen für Tickets konfigurierbar und speicherbar

\item alle Berichte in einem Extramenüpunkt.

\item Erstellung und Verwaltung einer Wissensdatenbank möglich
\end{itemize}

\noindent
Dieses Fazit der Analyse der verschiedenen Service Desk-Lösungen kann dazu genutzt werden, Verbesserungsmöglichkeiten im Service Desk Modul von GEBman 10 zu  bestimmen. 


   


 