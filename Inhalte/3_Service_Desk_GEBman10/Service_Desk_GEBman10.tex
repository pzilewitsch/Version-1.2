% !TEX root = Bachelorarbeit_Paul_Zilewitsch.tex

\section{Der Servicedesk in GEBman10}

\subsection{Aktuelle Umsetzung}

\subsection{Anforderungen für die Erweiterung}

\noindent
Qualität ist ein Maß für das Erfüllen von Anforderungen. Die Qualität des Servicedesk kann deshalb nur gesichert werden, wenn die Anforderungen möglichst genau definiert werden. Zu den zu erfüllen Anforderungen sollte aber noch festgehalten werden, welche Anforderungen nicht erfüllt werden sollen. Letzteres wird häufig nicht beachtet, ist jedoch ein wesentlicher Schritt für das Sicherstellen der Anforderungen. \footnote{Vorlesungsreihe Qualitätsmanagement: Hans-Jörg Günther, 6.Semester}

\colorbox{red}{hier muss noch einmal auf alle Punkte von 2. eingegangen werden, um abzugrenzen, was eine Anforderung ist und was eben nicht! }