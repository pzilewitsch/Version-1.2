% !TEX root = Bachelorarbeit_Paul_Zilewitsch.tex
\section{Fazit}

\noindent
In dieser Bachelorarbeit wurde das Service Desk Modul von GEBman 10 auf Verbesserungsmöglichkeiten untersucht. Hierfür wurde zunächst der Begriff Service Desk in einen Kontext gebracht und allgemeine Aufgaben festgehalten. Anschließend wurden verschiedene Service Desk-Softwarelösungen getestet und aus bestimmten Blickwinkeln analysiert. Daraus ließen sich Verbesserungsmöglichkeiten für die aktuelle Umsetzung des Service Desk Moduls herausfinden....\\ 

\noindent
Außerdem wurde eine Erweiterung der E-Mail Integration des Service Desk Moduls vorgenommen. Zuvor mussten die Exchange Web Services näher betrachtet werden. Danach konnte eine Konzipierung als Grundstein für die Implementierung gelegt werden. In der Umsetzung.....\\

\noindent
In diesem Punkt werden nun die Verbesserungsmöglichkeiten für das Service Desk Modul festgehalten und die Erweiterungsmöglichkeiten der Implementierung der E-Mail Integration erläutert. 



\subsection{Verbesserungsmöglichkeiten von dem Service Desk Modul}
\noindent
Bilder direkt in Text einfügen?
Welche Daten könnten hinzukommen?
Welche Features könnte man übernehmen, welche nicht?


\subsection{Erweiterungsmöglichkeiten der E-Mail Integration}
\noindent
Programmtechnisch:
Erweiterung der Benutzermöglichkeiten, Status ändern bzw. Meldung schließen. Über Betreff oder Link zu einem Webfrontend.


\subsection{Schlussbemerkung}