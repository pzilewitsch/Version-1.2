% !TEX root = Bachelorarbeit_Paul_Zilewitsch.tex
\section{Fazit}

\noindent
In dieser Bachelorarbeit wurde das Service Desk Modul von GEBman 10 auf Verbesserungsmöglichkeiten untersucht und eine Erweiterung der E-Mail Integration auf Basis von den Microsoft Exchange Web Services implementiert. Hierfür wurde zunächst der Begriff Service Desk erläutert, in einen Kontext gebracht und allgemeine Aufgaben festgehalten. Daraus hat sich ergeben, dass der Service Desk den Grundstein für eine gute Kommunikation zwischen Unternehmen und Kunden legt. Demnach ist das Arbeiten mit einer Service Desk-Softwarelösung Bestandteil des Arbeitsalltags und muss daher speziellen Anforderungen gerecht werden.\newline
Anschließend wurden im Punkt 3 verschiedene Service Desk-Softwarelösungen getestet und aus bestimmten Blickwinkeln analysiert. Diese Lösungen hatten sowohl Unterschiede als auch Gemeinsamkeiten in Hinblick auf die Funktionalitäten. Das lag nicht zuletzt an den unterschiedlichen Einsatzgebieten und gesetzten Schwerpunkten der einzelnen Lösungen.\newline
Durch die Betrachtung der aktuellen Umsetzung des Service Desk-Moduls im Punkt 4 bot einen Einblick in die Funktionalitäten eines speziell auf den Bereich Facility Management abgestimmten Service Desk. Mit Hilfe der Analyse aus Punkt 3 war es möglich, Verbesserungsmöglichkeiten für das Service Desk-Modul von GEBman 10 zu identifizieren. Außerdem wurden die Anforderung für die Erweiterung der E-Mail Integration in diesem Punkt festgehalten. \\

\noindent
Für die Umsetzung der Erweiterung mussten zuvor die Exchange Web Services näher betrachtet werden. Hierfür wurden die Grundlagen und die Funktionsweise von Microsoft Exchange kurz erläutert, um anschließend die Vorteile der zur Verfügung stehenden Web Services erklären zu können. Es stellte sich heraus, dass durch eine API der Web Services der Zugriff auf wichtige Elemente des Exchange Servers über Programmcode möglich war. Bevor jedoch eine direkte Implementierung in GEbman 10 vorgenommen wurde, musste eine Vorbetrachtung eine umfassende Modellierung vorgenommen werden. Dabei wurde auch auf Sicherheitsaspekte der Erweiterung eingegangen. Diese Modellierung im Punkt 6 erleichterte die anschließende Implementierung.\newline
Die Umsetzung wurde nicht nur auf die reine Implementierung des Programmcodes beschränkt. Es wurde aufgezeigt, wie die Managed API von Microsoft in das System eingebunden und genutzt wurde. Testfälle wurden entwickelt, um die wichtigsten Funktionalitäten sicherzustellen.  Außerdem wurde auf Probleme während der Implementierung eingegangen und Verbesserungsmöglichkeiten dargelegt.\\\\

\noindent
Diese Bachelorarbeit soll nun mit einem Ausblick auf die Verbesserungen des Service Desk-Moduls und auf den Einsatz der Erweiterung abgeschlossen werden. Das Service Desk-Modul von GEBman 10 könnte mit einigen Verbesserungen die aktuelle externe Softwarelösung SysAid ablösen und somit für die alltägliche Arbeit im Support eingesetzt werden. Dadurch könnten nicht nur Kosten reduziert werden, sondern es wäre auch möglich, das Service Desk-Modul weiter an die Bedürfnisse des Supports anzupassen. Erst nach einiger Zeit der Nutzung können sich Aspekte für die Verfeinerung herausstellen, die in dieser Arbeit nicht betrachtet werden konnten.\\

\noindent
Die E-Mail Integration könnte um die erwähnten Erweiterungsmöglichkeiten in Punkt 7.5 ergänzt und anschließend den Kunden präsentiert werden. Da es sich um eine prototypische Entwicklung handelt, ist darauf hinzuweisen, dass weitere Tests mit größeren Datenmengen ratsam wären. Es wurde mit dieser prototypischen Erweiterung dennoch der Grundstein für einen Einsatz als eigenständige Service Desk-Softwarelösung gelegt, die nicht mehr so stark an den Facility Management Bereich gebunden ist.


