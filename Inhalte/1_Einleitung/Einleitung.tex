% !TEX root = Bachelorarbeit_Paul_Zilewitsch.tex
\section{Einleitung}

%http://webuser.hs-furtwangen.de/~kaspar/seminar0405/Service_Desk.pdf

\subsection{Die Software GEBman 10 von der KMS Computer GmbH}
\noindent
Die IT ist ein ständiger Begleiter im Arbeitsalltag eines jeden Unternehmens. Sie erleichtert nicht nur die Arbeit mit großen Datenmengen sondern unterstützt auch wichtige Geschäftsprozesse in -und außerhalb des Unternehmens. Um einen möglichst effizienten Einsatz der IT zu ermöglichen, halten sich viele Unternehmen an bestimmte Best Practice Frameworks. Diese Leitfäden beinhalten unter anderem den Einsatz der IT im Bereich Kunden Service. Dabei übernimmt ein sogenannter Service Desk eine wichtige Funktion. Diese zentralen Anlaufstelle dient als Kommunikationsschnittstelle zwischen dem Kunden und der IT-Organisation. Viele Unternehmen setzen auf eine externe Service Desk-Softwarelösung, die dem Support eine bessere Kommunikation mit dem Kunden und ein effizienteres abarbeiten von Vorfällen ermöglicht.\\

\noindent
Die KMS Computer GmbH bietet ebenfalls eine Lösung eines Service Desk an, die jedoch auf den Bereich Facility Management spezialisiert ist. Das Unternehmen wurde 1990 gegründet und konzentriert sich heute vorrangig auf den Vertrieb von entwickelter Software im Bereich Computer-Aided Facility Management (CAFM). Computer-Aided bedeutet so viel wie \enquote{computergestützt}. Den Begriff Facility Management beschreibt Nävy präzise \enquote{als strategische Management-Disziplin, die die Analyse, Dokumentation und Optimierung aller kostenrelevanten Vorgänge rund um Gebäude und ihre Anlagen und Einrichtungen (Facilities) unter besonderer Berücksichtigung von Arbeitsplatz und Umfeld der Nutzer umfaßt.}\footnote{Vgl. \citeauthor{Naevy} (\citeyear{Naevy}), S. VII}\\

\noindent
Seit 2011 entwickelt und vertreibt die KMS Computer GmbH die webbasierte Software GEBman 10. Es handelt sich bei GEBman 10 um eine CAFM-Software für Kommunen, Industrie und Gebäudeverwalter. Übersichten geografischer Informationen oder die Analyse von Sachdaten können individuell auf die Kunden abgestimmt werden. Außerdem ist es ein Werkzeug zur Verwaltung und ein Arbeitsmittel zur Unternehmensführung oder der finanziellen Planung. Die Anwendung kann als Desktop-Installation, als interne Weblösung oder als Cloud-Lösung betrieben werden. Aber auch mobile Lösungen einzelner Module sind bereits in Verwendung und werden stetig weiterentwickelt. Es wird schon jetzt deutlich, dass GEBman 10 mannigfaltig ist und mit über 40 Modulen auch in vielen Branchen zum Einsatz kommt. Gerade in sehr speziellen Bereichen wie beispielsweise Außenbeleuchtung oder Baumverwaltung kann es zu den unterschiedlichsten und ungewöhnlichsten Problemen kommen. Eine Grundvoraussetzung  für die effiziente Lösung von Problemen, ist das Festhalten der genauen Vorkommnisse. Hierbei kann das Modul Service Desk in GEBman 10 durchaus hilfreich sein und alle erforderlichen Prozesse zur Problemlösung unterstützen.



\subsection{Das Modul Service Desk}
\noindent
Der Service Desk in GEBman 10 ist stark an die anderen Module gebunden und auf den Bereich Facility Management ausgelegt. Im Modul Service Desk ist es möglich, Meldungen für verschiedenste Objektarten aufzugeben. Ist Beispielsweise eine Außenbeleuchtung eines Gebäude ausgefallen, kann ein Benutzer eine Störmeldung bezüglich der Außenbeleuchtung aufgeben. Dabei wählt er das entsprechende Gebäude aus und trägt die genaue Problemstellung in die Meldung ein. Ein Techniker beispielsweise hat nun die Möglichkeit, auf die Störmeldung Einsicht zu nehmen und die defekte Außenbeleuchtung zu reparieren. Oder aber er fragt nach den genauen Ursache nach und antwortet somit auf die Störmeldung. Probleme und Vorfälle können durch den Service Desk genau spezifiziert und archiviert werden. Dadurch ist eine effizientere Lösung des Problems bei einem erneuten Auftreten möglich.\\


\subsection{Motivation der Arbeit}
\noindent
Die KMS Computer GmbH nutzt derzeit eine externe Service Desk-Software, um sämtliche Vorfälle in der Support-Abteilung zu bearbeiten und zu protokollieren. Ziel der Bachelorarbeit ist es, das Service Desk Modul in GEBman10 so anzupassen, dass auf eine externe Lösung verzichtet und das hauseigene Produkt eingesetzt werden kann. Hierfür muss das Modul Service Desk um eine E-Mail Integration erweitert werden. Mit dieser Erweiterung soll es möglich sein, über den E-Mail Verkehr auf Meldungen im Service Desk zu antworten oder neue Meldungen zu erstellen. Für die E-Mail Integration muss ein Konzept erstellt werden, um anschließend eine gute Implementierung zu erreichen.\newline
Des Weiteren sollen andere Service Desk-Lösungen analysiert werden, um Verbesserungsmöglichkeiten für das Service Desk Modul in GEBman10 zu identifizieren. Dadurch soll das Service Desk Modul vom Facility Management Bereich gelöst werden, wodurch dem Support ein komfortableres Arbeiten ermöglicht werden soll.\\


\subsection{Vorgehen bei der Arbeit}
\noindent
Zu Beginn wird im Punkt 2 der Begriff Service Desk in einen Kontext gebracht und allgemeine Anforderungen bestimmt. Durch eine Analyse verschiedener Service Desk - Softwarelösungen im Punkt 3 können Anregungen für mögliche Verbesserungen des Service Desk-Moduls gesammelt werden. Im Anschluss im Punkt 4 wird der gegenwärtige Service Desk in GEBman 10 durchleuchtet. Dabei werden die Erkenntnisse aus Punkt 3 dazu beitragen, Verbesserungsmöglichkeiten des Moduls zu benennen. Es werden außerdem die endgültigen Anforderungen an die E-Mail Integration beschrieben. Der Punkt 5 konzentriert sich auf den Microsoft Exchange Server, der einen wesentlichen Bestandteil der E-Maul Integration bildet. Auch wird an dieser Stelle die derzeitige Verwendung der Exchange Web Services in GEbman 10 näher betrachtet.\\

\noindent
Durch eine Analyse der bis dato errungenen Erkenntnisse, kann im Punkt 6 eine Konzipierung erfolgen. Hierzu werden verschiedene Modellierungen vorgenommen und auf Sicherheitsaspekte eingegangen. Bei der Umsetzung im Punkt 7 steht die implementierte Erweiterung des Moduls Service Desk im Vordergrund. Wichtige Methoden, Erfahrungen und Fehlschläge während der Umsetzung sollen hier erörtert werden. Abschließend kann dann im Punkt 8 eine Fazit gezogen werden. Ein Ausblick auf die Verbesserungen des Service Desk-Moduls in GEBman 10 werden diese Arbeit abrunden.